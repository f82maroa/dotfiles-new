


\chapter{El átomo de hidrógeno}
\pagestyle{simple} 
\newpage

\pagestyle{main}

Objetivo: Resolver el átomo de hidrógeno por completo\\
\\
Un átomo de hidrógeno es un protón más 1 electrón. \\
\textbf{Mundo clásico}: La energía será 
\[
	E=Ec_1+ Ec_2 + E_{potencial}
.\] 
La energía cinética será:
\[
E_c= \frac{p_1^2}{2m_1}+ \frac{p_2^2}{2m_2}	
.\] 
Sabemos que asociado a $\vec{p_1}= -i \hbar  \nabla_1 $ y $\vec{p_2}= -i\hbar \nabla_1$. Esto conmuta, es decir, $[p_1,p_2]=0$ Esto es debido a que las partículas son diferentes, esto nos permite hacer un arreglo en el mundo clásico antes de irnos al mundo cuántico. Reexpresamos la energía del sistema:
\[
	E=\frac{1}{2m}m_1v_1^2+\frac{1}{2}m_2v_2^2= \left( CM \right) + \left( c.rel \right) 
.\]  
Llamamos a los vectores posición $r_1$ y $r_2$ a la distancia de $m_1$ y $m_2$. Llamamos al centro de masas:
\[
	\vec{r_{cm}}=\frac{m_1r_1+m_2r_2}{m_1+m_2}
.\] 
y el vector 
\[
r=r_1-r_2
.\]
El problema lo haremos en función de $r$ y de $r_{cm}$. Si despejamos $r_1 $ y $r_2$ :
\[
	r_1= r_{cm}+ \frac{m_2}{M}r
.\] 
\[
	r_2=r_{cm} - \frac{m_1}{M}r
.\] 
La velocidad del señor $1$ será:
\[
v_1=\dot{r_1}= \dot{ri}_{cm}+ \frac{m_2}{M}\dot{r}
.\] 
El cuadrado de la velocidad es:
\[
	v_1^2=\left( v_1 \right) ^2=\dot{r}_{cm}^2+ \frac{m_2^2}{M^2} \dot{r}^2+\frac{2m_2}{M}\dot{r}_{cm}\cdot \dot{r}=V_{cm}^2+ \frac{m_2}{M} \mid \dot{r} \mid + \ldots 
.\] 
La diferencia con $v_2^2$ es clara, dará: \[
	v_2^2=V_{cm}^2 + \frac{m_1^2}{M^2}  \mid \dot{r}^2 \mid -\frac{2m_1}{M}\dot r_{cm}\cdot \dot{r}		
.\] 
La energía cinética será:
\[
	E_c= \frac{1}{2}m_1\left( V_{cm}^2+ \frac{m_2^2}{M^2} \mid \dot{r}^2 + \frac{2m_2}{M}\dot{r}_{cm} \cdot \dot{r}  \right) + E_2
.\] 
Esto es:
\[
	E_c= \frac{1}{2}M V_{cm}^2 + 
.\] 

