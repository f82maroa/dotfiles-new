\chapter{Chuletario}%
\label{sec:Chuletario}
\section{Tema 3: El momento angular en Mecánica Cuántica}%
\twocolumn
\textcolor{blue}{\textbf{a- Como actúa $\hat{L_+}$ y $  \hat{L_-}$ sobre los vectores $\bra{l,m}$} }
    \[
        \hat{L}_+\bra{l,m}=\hbar \sqrt{l\left( l+1 \right) -m \left( m+1 \right) } \bra{l,m+1}
    .\] 
    \[
        \hat{L}_-\bra{l,m}=\hbar \sqrt{l\left( l+1 \right) -m\left( m-1 \right) } \bra{l,m-1}
    .\] 
\textcolor{blue}{\textbf{b - Relaciones de $\hat{L}_\pm$ con $\hat{L}_x$ y $\hat{L}_y$}} 
\[
\hat{L}_+=\hat{L}_x + i \hat{L}_y
.\] 
\begin{center}
    y        
\end{center}
\[
\hat{L}_-=\hat{L}_x - i \hat{L}_y
.\] 
\textcolor{blue}{\textbf{c- Reglas de conmutación}}
\[
    \left[ \hat{L}_x . \hat{L}_y \right]= \hat{L}_x \hat{L}_y - \hat{L}_y \hat{L}_x= i \hbar \hat{L}_z
.\] 
 \[
\ldots
.\] 
\textcolor{red}{Meter las otras reglas de conmunación!!!!}

\subsection{Otras reglas de conmutación}
\begin{equation}
    [L_x,L_y]=0
\end{equation}
de conmutación son las siguientes:  

\begin{itemize}
    \item [1]: La primera   
            
\end{itemize}
