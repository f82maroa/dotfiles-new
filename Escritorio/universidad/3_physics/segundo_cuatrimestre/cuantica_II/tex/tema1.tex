\chapter{Vídeos de Javier García}
\section{Conmutadores en mecánica cuántica}
\textbf{¿Qué es un conmutador?}: $[A,B]=AB-BA$, \\
\\
donde $A,B$ son operadores, pero, ¿qué es un operador? es una acción que se hace sobre algo a la derecha \\
\\
En lo que sigue, trabajamos en la base de posiciones, es decir $\hat{x}= \cdot x$, donde $x$ es un escalar.\\
\\
En esta base, $\hat{p_x} = -i\hbar \frac{\partial  }{\partial x} $, $\hat{L_x}= y \frac{\partial  }{\partial z} -z \frac{\partial  }{\partial y} $, $\hat{L_y}=z \frac{\partial  }{\partial x} 
 x \frac{\partial  }{\partial z} $\ldots

 \subsection{Ejemplos de operadores}
 \[
     [\hat{x},\hat{x}]= 0
 .\] 
 Es decir, cualquier operador consigo mismo es 0
 \[
     [\hat{x},\hat{y}]\psi= xy\psi - yx \psi= xy\psi-xy\psi=0
 .\] 
 Todos los operadores posición consigo mismo son 0, es decir, 
 \[
 x_1=x; x_2=y; x_3=z
 .\] 
 \[
     [x_i,x_j]=0 ; 
 .\] 
 válida para todo $i,j$ \\
 \\
 Que pasa con $[\hat{x},\hat{p}_x]$ :
 \[
     [\hat{x},\hat{p}_x]\psi= \hat{x}\hat{p}\psi - \hat{p}\hat{x}\psi= x\left( -i\hbar \frac{\partial  }{\partial x}  \right) \psi + i\hbar \frac{\partial  }{\partial x} \left( x \psi \right) = -i\hbar \psi' + i\hbar \left( \psi + x\psi' \right) = -i\hbar x\psi'+i\hbar \psi +i\hbar x \psi'= i\hbar \psi
 .\] 
 

 


