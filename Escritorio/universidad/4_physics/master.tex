\documentclass[10pt,a4paper,twocolumn]{report}

\NewDocumentCommand{\myrule}{O{1pt} O{2pt} O{black}}{%
      \par\nobreak % don't break a page here
        \kern\the\prevdepth % don't take into account the depth of the preceding line
          \kern#2 % space before the rule
            {\color{#3}\hrule height #1 width\hsize} % the rule
              \kern#2 % space after the rule
                \nointerlineskip % no additional space after the rule
            }
%Carga los paquetes de paquetes.sty


\usepackage{sty/paquetes}
\usepackage{sty/comandos}
\usepackage{tikz,lipsum,lmodern}
\usepackage[most]{tcolorbox}
\usepackage[utf8]{inputenc}
\usepackage{blindtext}
\usepackage{multicol}
\usepackage{color}
\setlength{\columnseprule}{1pt}
\def\columnseprulecolor{\color{red}}
\makeatother
%======================================================================================================
%
%	Comienzo del documento
%
%======================================================================================================
\usepackage{fancyhdr}
\pagestyle{fancy}
\fancyhead[LO,RE]{\leftmark}
\fancyhead[LE,RO]{\includegraphics[scale=0.08]{img/ucm.png}}
\rfoot{Página \thepage\  de \pageref{LastPage} }
\cfoot{}


\begin{document}

%	Página del título

\begin{titlepage}
    \begin{center}
        \textbf{\LARGE Universidad Complutense de Madrid}\\[0.5cm] 
        \textbf{\large Facultad de Ciencias Físicas}\\[0.2cm]
        \vspace{20pt}\includegraphics[width=5cm]{img/ucm.png}\\
        \par
        \vspace{20pt}
          \textbf{\Large Grado en Física}\\
          \vspace{15pt}
          \myrule[1pt][7pt]
          \textbf{\LARGE  Apuntes de verano}\\
          \vspace{15pt}
          \textbf{\large Tercero de Física}\\
          \myrule[1pt][7pt]
          \vspace{35pt}
          \textbf{\large Nombre \hspace{100pt} Mail}\\
          \vspace{15pt}
          Álvaro Martín Romero \hspace{50pt} \textcolor{blue}{\textit{f82maroa@uco.es}} \\ 


          \vspace{70pt}  
          \textbf {\large Descripción:}\\[0.2cm]
          \Large {Resumen de asignaturas estudiadas en verano de 2021}\\[0.1cm]
      \end{center}

      \par
      \vfill
      \begin{center}
          \textbf{Dedicado a mi neni que me ha apoyado mucho este curso. Te quiero .}\\
      \end{center}

  \end{titlepage}

  
  \thispagestyle{fancy}



%	Algunas cosas tienen que ir antes del índice


%	Crea el índice


\tableofcontents

%\subfile{tex/prefacio}

%	Reinicia la numeración de páginas y la hace en números arábigos
% asignaturas


% 	Chuletario

%\chapter{ Electromagnetismo}
\section{Ecuaciones de Maxwell}%


hola


%\input{tex/circuitos}
%\chapter{Física CuánticaI}%
\section{Tema 1: Historia Física Cuántica}%

%	Comienza la inclusión de cada tema
\chapter{Electromagnetismo I}
\section{Tema 1: Campo eléctrico}
%	Tema 1: Campo eléctrico
\[
	\vec{E}\left( \vec{r} \right) =\frac{q}{4\pi \epsilon_0}\sum_{n=1}^{N} \frac{\vec{r}-\vec{r'_i}}{ \mid \vec{r}-\vec{r'_i} \mid ^3}
.\] 

%\chapter{Mecánica De Medios Continuos}

%\input{tex/estadistica.tex}
%\input{tex/cuanticaI.tex}
\appendix
\end{document}

